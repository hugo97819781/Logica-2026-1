%%%%%%%%%%%%%%%%%%%%%
%%% F O R M A T O %%%
%%%%%%%%%%%%%%%%%%%%%
    \documentclass[letterpaper,DIV=15,headsepline,12pt]{scrartcl}
    \usepackage[spanish,mexico,shorthands=off,es-lcroman]{babel}
%%%%%%%%%%%%%%%%%%%%%%%
%%% P A Q U E T E S %%%
%%%%%%%%%%%%%%%%%%%%%%%
    \usepackage{mathtools}
    \usepackage{amsthm}
    \usepackage{ifthen}
    \usepackage{truthtable} %para crear tablas de verdad
    \usepackage[x11names, table]{xcolor}
    \usepackage[most]{tcolorbox}
    \usepackage{multicol}
%%%%%%%%%%%%%%%%%%%%%
%%% F U E N T E S %%%
%%%%%%%%%%%%%%%%%%%%%
    \usepackage{fontspec}
    \usepackage{unicode-math}
    \setmainfont{GFSArtemisia.otf}[
        Path = fonts/GFSArtemisia/ ,
        BoldFont = GFSArtemisiaBold.otf,
        ItalicFont = GFSArtemisiaIt.otf,
        BoldItalicFont = GFSArtemisiaBoldIt.otf
    ]
    \setmathfont{STIXTwoMath-Regular.ttf}[
        Path = fonts/StixTwo/
    ]
%%%%%%%%%%%%%%%%%%%%%%%
%%% E S T É T I C A %%%
%%%%%%%%%%%%%%%%%%%%%%%
    \usepackage{scrlayer-scrpage}
        \clearpairofpagestyles
        \ihead{\footnotesize \textit{Lógica Matemática I}}
        \ohead{\footnotesize \textit{2026-1}}
        \cfoot{\normalfont\thepage}
        \addtokomafont{title}{\bfseries \rmfamily}
        \setlength{\headsep}{5pt}
        \newpairofpagestyles{beginstyle}{
            \clearpairofpagestyles
            \KOMAoptions{headsepline=false}
            \cfoot{\footnotesize \pagemark}
            \cfoot{\normalfont\thepage}
        }
        \addtokomafont{section}{\raggedleft\rmfamily}
        \addtokomafont{subsection}{\raggedleft\rmfamily}
        \addtokomafont{subsubsection}{\raggedleft\rmfamily}
        \setlength{\headsep}{8pt}
        \setlength{\footskip}{20pt}
        \setlength{\textheight}{600pt}
    \usepackage[shortlabels]{enumitem}
    \setenumerate[1]{label=\MakeLowercase{\roman*}), ref=\roman*}
    \setenumerate[2]{label=\MakeLowercase{\alph*}), ref=\alph*}
%%%%%%%%%%%%%%%%%%%%%%%
%%% C O M A N D O S %%%
%%%%%%%%%%%%%%%%%%%%%%%
    \newcommand{\mathds}[1]{\mathbb{#1}}
    \renewcommand{\emptyset}{\varnothing}
%%%%%%%%%%%%%%%%%%%%%%%
%% D O C U M E N T O %%
%%%%%%%%%%%%%%%%%%%%%%%
\begin{document}
    \thispagestyle{beginstyle}
    \begin{center}
        {\fontsize{30}{60}\rmfamily \textbf{Lista de Ejercicios}} \\ \vspace{.2cm}
        Lógica Matemática I, 2026-1
    \end{center}
    \begin{flushright}
        \footnotesize \hfill Profesor: Fernando Javier Nuñez Rosales.\\
        \hfill Ayudante: Hugo Víctor García Martínez.
    \end{flushright}

    \section*{Preliminares conjuntistas}
    \begin{enumerate}
        \item Demuestra que el conjunto de los subconjuntos finitos de números naturales, es decir es numerable.
        \item Un número se llama algebraico si es la raíz de un polinomio con coeficientes enteros. Muestra que el conjunto de números algebraicos es contable.
        \item Pruebe que que hay números reales que no son algebraicos.
        \item Sean $X$ un conjunto y $\{A_n : n\in\mathds{N} \}$ una familia de subconjuntos de $X$ tales que para cualesquiera $n,m\in\mathds{N}$ si $n<m$, entonces $A_n\subseteq A_m$. Si $k$ es un entero positivo y
        \begin{displaymath}
            a_1,\ldots a_k\in \bigcup_{n\in \mathds{N}}A_n\text{,}
        \end{displaymath}
        entonces existe un entero $l$ tal que $ a_1,\ldots a_k\in A_l$.
        \item De un contraejemplo para el enunciado anterior cuando se considera una cantidad infinita de elementos de la unión.
        \item Sean $X,Y$ conjuntos y $f:X \to Y$ cualquier función. Para cada $x \in X$, defínase $A_x:=\{y \in X : f(y)=f(x)\}$ y sea $C:=\{A_x : x \in A\}$.
        \begin{enumerate}
            \item Muestre que $\hat{f}:=\{(A_x,f(x)) : x \in X\}$, es función.
            \item Demuestre que $\hat{f}:C \to f[X]$ es biyectiva.
            \item Concluya que toda función es composición de una función sobreyectiva, una función biyectiva y una función inyectiva.
        \end{enumerate}
    \end{enumerate}

    \newpage
    \section*{Conjunto B-F-Induntivos}
        \begin{enumerate}
            \item Construye los polinomios con coeficientes en los reales como conjunto B-F-inductivo.
            \item Construye los números enteros como conjunto mínimo B-F-Inductivo.
            \item ¿Cuantos elementos tiene un conjunto B-F-Inductivo?
            \item Sea $X:=P(\mathds{R})$ el conjunto potencia de los reales. ¿Existe una operación $f:X \to X$? tal que $X$ es el mínimo conjunto $\{\emptyset\}$-$\{f\}$ inductivo?
            \item Construye a $\overline{L}$ como como conjunto B-F-inductivo.
            \item Demuestre que si cierta propiedad, $\mathcal{P}(x)$ de números enteros, se cumple para cada primo $p$ y se preserva bajo productos, entonces se cumple para todo número entero mayor o igual que $2$.
            \item Sean $U$ un conjunto, $B \subseteq U$ y $F$ un conjunto de operaciones en $U$. Si $X:=B^*$ es libremente generado por $B$, ¿es posible que $X$ sea finito?. Argumente su resupuesta.
            \item Demuestre que:
            \begin{enumerate}
                \item Si $U=\mathds{N}$, $B=\{0\}$ y $F=\{s\}$ (con $s:\mathds{N} \to \mathds{N}$ la función sucesor); entonces $\mathds{N}=B^*$ es libremente generado.
                \item Si $U=\mathds{Z}$, $B=\{0\}$ y $F=\{s,p\}$ (con $s,p:\mathds{Z} \to \mathds{Z}$ las funciones sucesor y predecesor, respectivamente); entonces $\mathds{Z}=B^*$ NO es libremente generado.
                \item Existen un racional $q_0 \in \mathds{Q}$ y una operación $g:\mathds{Q} \to \mathds{Q}$ tales que $\mathds{Q}$ es libremente generado por $\{q_0\}$ y $\{g\}$.
            \end{enumerate}
            \item Utilizando el Teorema de Recursión para Naturales, demuestre la existencia del factorial. Esto es, pruebe que existe una función $h:\mathds{N} \to \mathds{N}$ de modo que $h(0)=1$; y, para cada $n \in \mathds{N}$ se cumpla $h(n+1)=(n+1)h(n)$.
        \end{enumerate}
    
    \newpage
    \section*{Lenguajes y sistemas formales}
        \begin{enumerate}
            \item Da tres ejemplos de alfabetos finitos, tres ejemplos de alfabetos numerables y tres ejemplos de alfabetos incontables.
        \end{enumerate}
    
    \newpage
    \section*{Procedimientos efectivos}
        \begin{enumerate}
            \item Muestra que un subconjunto $X\subseteq \mathds{N}$ es decidible si y solo si $\mathds{N}-X$ también lo es.
            \item Sea $X\subseteq \mathds{N}$. Prueba que si $X$ y $\mathds{N}-X$ son efectivamente numerables, entonces $X$ es decidible.
            \item Muestra que el conjunto de las tautologías es decidible.
        \end{enumerate}
    
    \newpage
    \section*{Álgebras de Boole}
        \begin{enumerate}
            \item Prueba que $\{[A]_\equiv : A\in\overline{L}\}$ puede ser dotada de estructura de álgebra de Boole.
        \end{enumerate}
    
    \newpage
    \section*{Lenguaje proposicional}
        \begin{enumerate}
            \item Menciona tres proposiciones y tres paradojas.
            \item Muestra que no hay fórmulas de longitud 2, 3 ni 6, pero que cualquier otra longitud es posible.
            \item Muestra que en toda fórmula el número de lugares que usa algún conectivo binario más uno es el número de lugares ocupados por letras proposicionales.
            \item Muestra que la longitud de una fórmula en la que no ocurre el símbolo de negación es impar.
            \item Muestra que en una fórmula en la que no aparece la negación más de una cuarta parte de los símbolos de la fórmula son letras proposicionales.
            \item Define la función que a cada fórmula proposicional le asocia el conjunto de las letras proposicionales que ocurren en ella.
            \item Define recursivamente la función que a cada fórmula le asocia el número de conectivos que ocurren en ella.
            \item Demuestra que la inducción por número de conectivos para $\overline{L}$ es un método valido.
        \end{enumerate}
    
    \newpage
    \section*{La semántica de la lógica proposicional}
        \begin{enumerate}
            \item Muestra que todas las tautologías son equivalentes entre si.
            \item Muestra que todas las falacias son equivalentes entre si
            \item Muestra que la equivalencia lógica módulo un conjunto de fórmulas es una relación de equivalencia entre fórmulas.
            \item Cuantas clases de equivalencia tiene la relación de equivalencia lógica.
            
            \item Sean $\alpha, \beta, \gamma$ fórmulas cualesquiera. Demuestre un inciso de cada una de las siguientes columnas:
            \begin{multicols}{3}
            \begin{enumerate}
                \item $\alpha \vDash \alpha \lor \beta$.
                \item $\alpha \land (\alpha \to \beta) \vDash \beta $.
                \item $\beta \to (\alpha \land \lnot \alpha) \vDash (\alpha \land \lnot \alpha) \to \beta$.
            \end{enumerate}
            \end{multicols}
            Encuentre además fórmulas $\alpha$ y $\beta$ de modo que:
            \begin{multicols}{3}
            \begin{enumerate}
                \item $\alpha \lor \beta \not\vDash \alpha$.
                \item $\beta \not \vDash \alpha \land (\alpha \to \beta)$.
                \item $(\alpha \land \lnot \alpha) \to \beta \not\vDash \beta \to (\alpha \land \lnot \alpha)$.
            \end{enumerate}
            \end{multicols}
            
            \item Sean $\alpha, \beta, \gamma$ fórmulas cualesquiera. Demuestre un inciso de cada una de las siguientes columnas:
            \begin{multicols}{3}
            \begin{enumerate}
                \item $\alpha \equiv \lnot ( \lnot \alpha )$.
                \item $\alpha \equiv \alpha \land \alpha$.
                \item $\alpha \equiv \alpha \lor \alpha$.
                
                \item $\alpha \to \beta = \lnot \alpha \lor \beta $.
                \item $\lnot(\alpha \land \beta) \equiv \lnot \alpha \lor \lnot \beta$.
                \item $\lnot( \alpha \to \beta) \equiv \alpha \land \lnot \beta$.
                
                \item $\alpha \to (\beta \to \gamma) \equiv (\alpha \land \beta) \to \gamma $.
                \item $\lnot \alpha \equiv \alpha \to (\beta \land \lnot \beta)$.
                \item $\alpha \lor \lnot \alpha \equiv \alpha \to (\gamma \to \alpha)$.
            \end{enumerate}
            \end{multicols}
    
            \item Sean $\alpha$ y $\beta$ fórmulas. Demuestre que $\alpha \to \beta \vDash \beta \to \alpha$ si y sólo si $\beta \to \alpha$ es tautología. Concluya utilizando lo anterior que $\alpha \to \beta \equiv \beta \to \alpha$ si y sólo si $\alpha \equiv \beta$.
            
            \item De un ejemplo de una fómrmula $\alpha$ que no sea contradicción ni tautología. Pruebe además que cualquier contradicción implica lógicamente a $\alpha$ y que de $\alpha$ se sigue cualquier tautología.
            
            \item Da un ejemplo de conjunto de fórmulas $\Sigma$ de tal modo que $\equiv_\Sigma$ tenga una cantidad finita de clases diferente a 1.
            \item Da tres ejemplos de tautologías (con la demostración de que lo son).
            \item Prueba que si $\Sigma$ es un conjunto de fórmulas y $\alpha ,\beta\in \overline{L}$, entonces
                \begin{displaymath}
                    \Sigma\cup\{\alpha\}\vDash\beta\text{ si y solo si }\Sigma\vDash \alpha\longrightarrow\beta\text{.}
                \end{displaymath}
            \item Muestra que dos fórmulas $A$ y $B$ son equivalente si y solo si $A\longleftrightarrow B$ es tautología.
            \item Prueba o refuta que las siguientes afirmaciones donde $\Sigma$ es un conjunto de fórmulas y $B ,A\in \overline{L}$ .
                \begin{enumerate}
                    \item Si $\Sigma\vDash A$ o $\Sigma\vDash B$, entonces $\Sigma\vDash A\vee B$.
                    \item Si  $\Sigma\vDash A\vee B$, entonces $\Sigma\vDash A$ o $\Sigma\vDash B$.
                    \item Si $\Sigma\vDash A$ y $\Sigma\vDash B$, entonces $\Sigma\vDash A\And B$.
                    \item Si $\Sigma\vDash \neg A$, entonces $\Sigma\nvDash A$.
                    \item Si $\Sigma\nvDash A$, entonces $\Sigma\vDash\neg A$.
                \end{enumerate}
            \item Sea $(A_n)_n$ una sucesión de fórmulas proposicionales. Para cada fórmula $B\in\overline{L}$, se define $B^*$ como la fórmula que resulta de sustituir en $B$ cada ocurrencia de la letra proposicional $P_j$ por la fórmula $A_j$.
                \begin{enumerate}
                    \item Defina recursivamente la asignación $B\mapsto B^*$; y
                    \item Prueba que si $v$ es una asignación de verdad para las letras proposicionales y se define la asignación de verdad para las letras proposiciones $u$ tal que $u(P_n)=\overline{v} (A_n)$, entonces $\overline{u} (B)=\overline{v}(B^*)$.
                \end{enumerate} 
            \item Sean $A, B\in\overline{L}$. Para cada fórmula $F\in\overline{L}$ sea $F_A^B$ la fórmula que resulta de sustituir en $F$ cada ocurrencia de la fórmula $A$ por la fórmula $B$.
                \begin{enumerate}
                    \item Define recursivamente $F_A^B$; y
                    \item Prueba que si $A\equiv B$, entonces $F_A^B\equiv F$.
                \end{enumerate}
            \item Sea $A\in\overline{L}$ un fórmula en la que solo ocurren los conectivos $\vee ,\And$ y $\neg$. Tómese $A^*$ como el resultado de \textbf{intercambiar} $\vee$ y $\And$; y de sustituir la ocurrencia de cada letra proposicional por su negación. Prueba que $A^*\equiv \neg A$.
            \item Prueba que los siguiente conjuntos de conectivos son completos $\{\vee, \neg\}$ y $\{\And ,\neg\}$.
        \end{enumerate}
    
    \newpage
    \section*{La sintaxis de la lógica proposicional}
        \begin{enumerate}
            \item Da una deducción en el CProp de las siguientes fórmulas.No se puede usar el teorema de la deducción.
                \begin{enumerate}
                    \item $((A\longrightarrow (B\longrightarrow C))\longrightarrow (B\longrightarrow (A\longrightarrow C)))$
                    \item $((A\longrightarrow B)\longrightarrow ((B\longrightarrow C)\longrightarrow (A\longrightarrow C))$
                    \item  $(\neg (\neg A))\longrightarrow A$
                    \item $A\longrightarrow (\neg (\neg A))$
                    \item  $((A\longrightarrow B)\longrightarrow ((\neg B)\longleftarrow(\neg A)))$
                \end{enumerate}
        \end{enumerate}
    
    \newpage
    \section*{Teorema de Compacidad}
        \begin{enumerate}
            \item Sea $\Sigma\subseteq\overline{L}$ y $A\in\overline{L}$. Si $\Sigma$ es finitamente satisfacible, entonces $\Sigma\cup\{A\}$ es finitamente satisfacible o $\Sigma\cup\{\neg A\}$ es finitamente satisfacible.
            \item  Sea $\Delta\subseteq\overline{L}$ tal que,
                    \begin{itemize}
                        \item $\Delta$ es finitamente satisfacible; y
                        \item Para toda $A\in\overline{L}$ se tiene que $A\in \Delta$ o $\neg A\in\Delta$.
                    \end{itemize}
                    Prueba que $\Delta$ es satisfacible.
        \end{enumerate}
    
    \newpage
    \section*{Teorema de completud-correctud}
        \begin{enumerate}
            \item Pruebe que el CProp es consistente, es decir, no existe $A\in \Bar{L}$ tal que $\vdash A$ y $\vdash (\neg A)$.
            \item Prueba que el teorema de compacidad es equivalente al teorema de completud correctud.
            \item Prueba que el teorema de compacidad es equivalente a que para cualquier $\Sigma\subseteq \Bar{L}$ y cualquier fórmula $A$, $\Sigma\vDash A$ si y solo si existe un subconjunto finito $\Sigma_0\subseteq\Sigma$ tal que $\Sigma_0\vDash A$.
        \end{enumerate}

\end{document}